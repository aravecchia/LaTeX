\documentclass[final]{beamer}
\usepackage[size=a0,scale=1.0]{beamerposter} % A0 paisagem
\usepackage{tikz}
\setbeamertemplate{navigation symbols}{}

\begin{document}
	\begin{frame}[t]
		
		% === Moldura externa e selo ===
		\begin{tikzpicture}[remember picture, overlay]
			
			% Moldura com 1 cm de margem
			\draw[line width=5pt]
			([shift={(1cm,-1cm)}]current page.north west)
			rectangle
			([shift={(-1cm,1cm)}]current page.south east);
			
			% Selo A4 (210 x 297 mm), no canto inferior direito
			\draw[line width=0.8pt]
			([shift={(-1cm,1cm)}]current page.south east)
			rectangle ++(-210mm,297mm);
			
		\end{tikzpicture}
		
		% === Conteúdo interno do selo ===
		\begin{tikzpicture}[remember picture, overlay, x=1mm, y=1mm]
			% Origem posicionada no canto inferior direito, recuando 1cm + 210mm para esquerda
			\begin{scope}[shift={(1189-10-210,10)}]  % 1189mm = largura da folha A0
				
				% Referência externa do selo (210 x 297 mm)
				\draw[gray!50] (0,0) rectangle (210,297);
				
				% Faixas horizontais internas
				\draw (0,0) rectangle (210,30);    % Faixa inferior
				\draw (0,30) rectangle (210,60);   % Faixa de título
				\draw (0,60) rectangle (210,90);   % Faixa auxiliar
				
				% Divisões verticais na faixa inferior
				\draw (50,0) -- (50,30);
				\draw (100,0) -- (100,30);
				\draw (150,0) -- (150,30);
				
				% Rótulos
				\node[anchor=west, font=\tiny] at (2,5) {Projeto};
				\node[anchor=west, font=\tiny] at (52,5) {Escala};
				\node[anchor=west, font=\tiny] at (102,5) {Data};
				\node[anchor=west, font=\tiny] at (152,5) {Folha};
				
				% Título do desenho
				\node[anchor=west, font=\bfseries\large] at (5,40) {TÍTULO DO DESENHO TÉCNICO};
				
			\end{scope}
		\end{tikzpicture}
		
	\end{frame}
\end{document}
